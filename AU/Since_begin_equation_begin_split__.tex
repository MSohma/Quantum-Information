Since
\begin{equation}
\begin{split}
\Delta^{-it}\ell(A)\Delta^{it}(X)&=\Delta^{-it}(A\rho^{it}X\rho^{-it})=\rho^{-it}A\rho^{it}X\rho^{-it}\rho^{it}\\
                                 &=\rho^{-it}A\rho^{it}X=\ell (\rho^{-it}A\rho^{it})(X),
\end{split}
\end{equation}
we have 
$$
\Delta^{-it}\ell(A)\Delta^{it}=\ell(\rho^{-it}A\rho^{it}).
$$
On the other hand,
we have
$$
J\ell(A)J(X)=J(AX^\ast)=(AX^\ast)^\ast=XA^\ast=r(A^\ast)(X)
$$
Thus we obtain the main results of Tomita-Takesaki theory in our case: 
\begin{equation}
\begin{split}
\Delta^{-it}\mathfrak{M}\Delta^{it}=\mathfrak{M},\\
          J\mathfrak{M}J=\mathfrak{M}^{\prime}.
\end{split}
\end{equation}
Remark that the above discussion can be easily extend to the case where 
$\mathfrak{N}=M_{m_1}(\mathbb{C})\oplus \cdots \oplus M_{m_n}(\mathbb{C})$.
The proof of Tomita-Takesaki theory for a finite dimensional von Neumann algebra is given in the Appendix.



