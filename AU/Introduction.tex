\section{Introduction}
This paper reviews results about  modular operators \cite{Longo_1978} and  commutation operators \cite{Holevo_1977} in the simplest case.
We mainly deal with the von Neumann algebra $\mathfrak{N}=M_n(\mathbb{C})$, 
which is the ensemble of $n\times n$ complex matrices and can be considered as the algebra $\mathfrak{B}({\cal H})$ of bounded linear operators on 
the Hilbert space ${\cal H}=\mathbb{C}^{n}$ with the inner product $(x,y)=\bar{x}^Ty$. 
Let $\rho$ be a non-degenerated density operator and 
$\omega$ be the corresponding normal state given by $\omega(A)=\mbox{Tr}\rho A, A\in \mathfrak{N}$.
We regard $\mathfrak{N}$ as a Hilbert space with the inner product
\begin{equation}
\label{innerP}
\langle A, B \rangle =\frac{1}{2}\omega(BA^{\ast}+A^{\ast}B),
\end{equation}
and denote it by $\mathfrak{H}$.
In the quantum theory we consider additional bilinear form on $\mathfrak{H}$ 
\begin{equation}\label{Bform}
[A,B]=i\omega(A^{\ast}B-BA^{\ast}).
\end{equation}
and  obtain fundamental inequalities 
$$
\langle X, X\rangle \geq \pm \frac{i}{2}[X,X],
$$
which yield  the uncertainty relation of the most general form \cite{Holevo_1977}.
We define a commutation operator $\mathfrak{D}$ so that it satisfies 
\begin{equation}\label{Copr}
[A,X]=\langle A, \mathfrak{D}X\rangle.
\end{equation}
The operator $\mathfrak{D}$, firstly introduced by Holevo \cite{Holevo_1977}, plays an important role in the non-commutative statistical theory. From Eq. (\ref{Bform}) it holds that 
\begin{equation}
1\pm \frac{i}{2}\mathfrak{D}\geq 0.
\end{equation}

