
%% bare_jrnl.tex
%% V1.3
%% 2007/01/11
%% by Michael Shell
%% see http://www.michaelshell.org/
%% for current contact information.
%%
%% This is a skeleton file demonstrating the use of IEEEtran.cls
%% (requires IEEEtran.cls version 1.7 or later) with an IEEE journal paper.
%%
%% Support sites:
%% http://www.michaelshell.org/tex/ieeetran/
%% http://www.ctan.org/tex-archive/macros/latex/contrib/IEEEtran/
%% and
%% http://www.ieee.org/



% *** Authors should verify (and, if needed, correct) their LaTeX system  ***
% *** with the testflow diagnostic prior to trusting their LaTeX platform ***
% *** with production work. IEEE's font choices can trigger bugs that do  ***
% *** not appear when using other class files.                            ***
% The testflow support page is at:
% http://www.michaelshell.org/tex/testflow/


%%*************************************************************************
%% Legal Notice:
%% This code is offered as-is without any warranty either expressed or
%% implied; without even the implied warranty of MERCHANTABILITY or
%% FITNESS FOR A PARTICULAR PURPOSE!
%% User assumes all risk.
%% In no event shall IEEE or any contributor to this code be liable for
%% any damages or losses, including, but not limited to, incidental,
%% consequential, or any other damages, resulting from the use or misuse
%% of any information contained here.
%%
%% All comments are the opinions of their respective authors and are not
%% necessarily endorsed by the IEEE.
%%
%% This work is distributed under the LaTeX Project Public License (LPPL)
%% ( http://www.latex-project.org/ ) version 1.3, and may be freely used,
%% distributed and modified. A copy of the LPPL, version 1.3, is included
%% in the base LaTeX documentation of all distributions of LaTeX released
%% 2003/12/01 or later.
%% Retain all contribution notices and credits.
%% ** Modified files should be clearly indicated as such, including  **
%% ** renaming them and changing author support contact information. **
%%
%% File list of work: IEEEtran.cls, IEEEtran_HOWTO.pdf, bare_adv.tex,
%%                    bare_conf.tex, bare_jrnl.tex, bare_jrnl_compsoc.tex
%%*************************************************************************

% Note that the a4paper option is mainly intended so that authors in
% countries using A4 can easily print to A4 and see how their papers will
% look in print - the typesetting of the document will not typically be
% affected with changes in paper size (but the bottom and side margins will).
% Use the testflow package mentioned above to verify correct handling of
% both paper sizes by the user's LaTeX system.
%
% Also note that the "draftcls" or "draftclsnofoot", not "draft", option
% should be used if it is desired that the figures are to be displayed in
% draft mode.
%
\documentclass[journal]{IEEEtran}
%
% If IEEEtran.cls has not been installed into the LaTeX system files,
% manually specify the path to it like:
% \documentclass[journal]{../sty/IEEEtran}





% Some very useful LaTeX packages include:
% (uncomment the ones you want to load)


% *** MISC UTILITY PACKAGES ***
%
\usepackage{ifpdf}
% Heiko Oberdiek's ifpdf.sty is very useful if you need conditional
% compilation based on whether the output is pdf or dvi.
% usage:
% \ifpdf
%   % pdf code
% \else
%   % dvi code
% \fi
% The latest version of ifpdf.sty can be obtained from:
% http://www.ctan.org/tex-archive/macros/latex/contrib/oberdiek/
% Also, note that IEEEtran.cls V1.7 and later provides a builtin
% \ifCLASSINFOpdf conditional that works the same way.
% When switching from latex to pdflatex and vice-versa, the compiler may
% have to be run twice to clear warning/error messages.



\usepackage{graphicx}
\usepackage[space]{grffile}
\usepackage{latexsym}
\usepackage{textcomp}
\usepackage{longtable}
\usepackage{multirow,booktabs}
\usepackage{amsfonts,amsmath,amssymb}
\usepackage{url}
\usepackage{hyperref}
\hypersetup{colorlinks=false,pdfborder={0 0 0}}
% You can conditionalize code for latexml or normal latex using this.
\newif\iflatexml\latexmlfalse
\providecommand{\tightlist}{\setlength{\itemsep}{0pt}\setlength{\parskip}{0pt}}%

\usepackage[utf8]{inputenc}
\usepackage[english]{babel}




% *** CITATION PACKAGES ***
%
\usepackage{cite}
% cite.sty was written by Donald Arseneau
% V1.6 and later of IEEEtran pre-defines the format of the cite.sty package
% \cite{} output to follow that of IEEE. Loading the cite package will
% result in citation numbers being automatically sorted and properly
% "compressed/ranged". e.g., [1], [9], [2], [7], [5], [6] without using
% cite.sty will become [1], [2], [5]--[7], [9] using cite.sty. cite.sty's
% \cite will automatically add leading space, if needed. Use cite.sty's
% noadjust option (cite.sty V3.8 and later) if you want to turn this off.
% cite.sty is already installed on most LaTeX systems. Be sure and use
% version 4.0 (2003-05-27) and later if using hyperref.sty. cite.sty does
% not currently provide for hyperlinked citations.
% The latest version can be obtained at:
% http://www.ctan.org/tex-archive/macros/latex/contrib/cite/
% The documentation is contained in the cite.sty file itself.


% *** MATH PACKAGES ***
%
%\usepackage[cmex10]{amsmath}
% A popular package from the American Mathematical Society that provides
% many useful and powerful commands for dealing with mathematics. If using
% it, be sure to load this package with the cmex10 option to ensure that
% only type 1 fonts will utilized at all point sizes. Without this option,
% it is possible that some math symbols, particularly those within
% footnotes, will be rendered in bitmap form which will result in a
% document that can not be IEEE Xplore compliant!
%
% Also, note that the amsmath package sets \interdisplaylinepenalty to 10000
% thus preventing page breaks from occurring within multiline equations. Use:
%\interdisplaylinepenalty=2500
% after loading amsmath to restore such page breaks as IEEEtran.cls normally
% does. amsmath.sty is already installed on most LaTeX systems. The latest
% version and documentation can be obtained at:
% http://www.ctan.org/tex-archive/macros/latex/required/amslatex/math/





% *** SPECIALIZED LIST PACKAGES ***
%
%\usepackage{algorithmic}
% algorithmic.sty was written by Peter Williams and Rogerio Brito.
% This package provides an algorithmic environment fo describing algorithms.
% You can use the algorithmic environment in-text or within a figure
% environment to provide for a floating algorithm. Do NOT use the algorithm
% floating environment provided by algorithm.sty (by the same authors) or
% algorithm2e.sty (by Christophe Fiorio) as IEEE does not use dedicated
% algorithm float types and packages that provide these will not provide
% correct IEEE style captions. The latest version and documentation of
% algorithmic.sty can be obtained at:
% http://www.ctan.org/tex-archive/macros/latex/contrib/algorithms/
% There is also a support site at:
% http://algorithms.berlios.de/index.html
% Also of interest may be the (relatively newer and more customizable)
% algorithmicx.sty package by Szasz Janos:
% http://www.ctan.org/tex-archive/macros/latex/contrib/algorithmicx/




% *** ALIGNMENT PACKAGES ***
%
%\usepackage{array}
% Frank Mittelbach's and David Carlisle's array.sty patches and improves
% the standard LaTeX2e array and tabular environments to provide better
% appearance and additional user controls. As the default LaTeX2e table
% generation code is lacking to the point of almost being broken with
% respect to the quality of the end results, all users are strongly
% advised to use an enhanced (at the very least that provided by array.sty)
% set of table tools. array.sty is already installed on most systems. The
% latest version and documentation can be obtained at:
% http://www.ctan.org/tex-archive/macros/latex/required/tools/


%\usepackage{mdwmath}
%\usepackage{mdwtab}
% Also highly recommended is Mark Wooding's extremely powerful MDW tools,
% especially mdwmath.sty and mdwtab.sty which are used to format equations
% and tables, respectively. The MDWtools set is already installed on most
% LaTeX systems. The lastest version and documentation is available at:
% http://www.ctan.org/tex-archive/macros/latex/contrib/mdwtools/


% IEEEtran contains the IEEEeqnarray family of commands that can be used to
% generate multiline equations as well as matrices, tables, etc., of high
% quality.


%\usepackage{eqparbox}
% Also of notable interest is Scott Pakin's eqparbox package for creating
% (automatically sized) equal width boxes - aka "natural width parboxes".
% Available at:
% http://www.ctan.org/tex-archive/macros/latex/contrib/eqparbox/





% *** SUBFIGURE PACKAGES ***
%\usepackage[tight,footnotesize]{subfigure}
% subfigure.sty was written by Steven Douglas Cochran. This package makes it
% easy to put subfigures in your figures. e.g., "Figure 1a and 1b". For IEEE
% work, it is a good idea to load it with the tight package option to reduce
% the amount of white space around the subfigures. subfigure.sty is already
% installed on most LaTeX systems. The latest version and documentation can
% be obtained at:
% http://www.ctan.org/tex-archive/obsolete/macros/latex/contrib/subfigure/
% subfigure.sty has been superceeded by subfig.sty.



%\usepackage[caption=false]{caption}
%\usepackage[font=footnotesize]{subfig}
% subfig.sty, also written by Steven Douglas Cochran, is the modern
% replacement for subfigure.sty. However, subfig.sty requires and
% automatically loads Axel Sommerfeldt's caption.sty which will override
% IEEEtran.cls handling of captions and this will result in nonIEEE style
% figure/table captions. To prevent this problem, be sure and preload
% caption.sty with its "caption=false" package option. This is will preserve
% IEEEtran.cls handing of captions. Version 1.3 (2005/06/28) and later
% (recommended due to many improvements over 1.2) of subfig.sty supports
% the caption=false option directly:
%\usepackage[caption=false,font=footnotesize]{subfig}
%
% The latest version and documentation can be obtained at:
% http://www.ctan.org/tex-archive/macros/latex/contrib/subfig/
% The latest version and documentation of caption.sty can be obtained at:
% http://www.ctan.org/tex-archive/macros/latex/contrib/caption/




% *** FLOAT PACKAGES ***
%
%\usepackage{fixltx2e}
% fixltx2e, the successor to the earlier fix2col.sty, was written by
% Frank Mittelbach and David Carlisle. This package corrects a few problems
% in the LaTeX2e kernel, the most notable of which is that in current
% LaTeX2e releases, the ordering of single and double column floats is not
% guaranteed to be preserved. Thus, an unpatched LaTeX2e can allow a
% single column figure to be placed prior to an earlier double column
% figure. The latest version and documentation can be found at:
% http://www.ctan.org/tex-archive/macros/latex/base/



%\usepackage{stfloats}
% stfloats.sty was written by Sigitas Tolusis. This package gives LaTeX2e
% the ability to do double column floats at the bottom of the page as well
% as the top. (e.g., "\begin{figure*}[!b]" is not normally possible in
% LaTeX2e). It also provides a command:
%\fnbelowfloat
% to enable the placement of footnotes below bottom floats (the standard
% LaTeX2e kernel puts them above bottom floats). This is an invasive package
% which rewrites many portions of the LaTeX2e float routines. It may not work
% with other packages that modify the LaTeX2e float routines. The latest
% version and documentation can be obtained at:
% http://www.ctan.org/tex-archive/macros/latex/contrib/sttools/
% Documentation is contained in the stfloats.sty comments as well as in the
% presfull.pdf file. Do not use the stfloats baselinefloat ability as IEEE
% does not allow \baselineskip to stretch. Authors submitting work to the
% IEEE should note that IEEE rarely uses double column equations and
% that authors should try to avoid such use. Do not be tempted to use the
% cuted.sty or midfloat.sty packages (also by Sigitas Tolusis) as IEEE does
% not format its papers in such ways.


%\ifCLASSOPTIONcaptionsoff
%  \usepackage[nomarkers]{endfloat}
% \let\MYoriglatexcaption\caption
% \renewcommand{\caption}[2][\relax]{\MYoriglatexcaption[#2]{#2}}
%\fi
% endfloat.sty was written by James Darrell McCauley and Jeff Goldberg.
% This package may be useful when used in conjunction with IEEEtran.cls'
% captionsoff option. Some IEEE journals/societies require that submissions
% have lists of figures/tables at the end of the paper and that
% figures/tables without any captions are placed on a page by themselves at
% the end of the document. If needed, the draftcls IEEEtran class option or
% \CLASSINPUTbaselinestretch interface can be used to increase the line
% spacing as well. Be sure and use the nomarkers option of endfloat to
% prevent endfloat from "marking" where the figures would have been placed
% in the text. The two hack lines of code above are a slight modification of
% that suggested by in the endfloat docs (section 8.3.1) to ensure that
% the full captions always appear in the list of figures/tables - even if
% the user used the short optional argument of \caption[]{}.
% IEEE papers do not typically make use of \caption[]'s optional argument,
% so this should not be an issue. A similar trick can be used to disable
% captions of packages such as subfig.sty that lack options to turn off
% the subcaptions:
% For subfig.sty:
% \let\MYorigsubfloat\subfloat
% \renewcommand{\subfloat}[2][\relax]{\MYorigsubfloat[]{#2}}
% For subfigure.sty:
% \let\MYorigsubfigure\subfigure
% \renewcommand{\subfigure}[2][\relax]{\MYorigsubfigure[]{#2}}
% However, the above trick will not work if both optional arguments of
% the \subfloat/subfig command are used. Furthermore, there needs to be a
% description of each subfigure *somewhere* and endfloat does not add
% subfigure captions to its list of figures. Thus, the best approach is to
% avoid the use of subfigure captions (many IEEE journals avoid them anyway)
% and instead reference/explain all the subfigures within the main caption.
% The latest version of endfloat.sty and its documentation can obtained at:
% http://www.ctan.org/tex-archive/macros/latex/contrib/endfloat/
%
% The IEEEtran \ifCLASSOPTIONcaptionsoff conditional can also be used
% later in the document, say, to conditionally put the References on a
% page by themselves.





% *** PDF, URL AND HYPERLINK PACKAGES ***
%
%\usepackage{url}
% url.sty was written by Donald Arseneau. It provides better support for
% handling and breaking URLs. url.sty is already installed on most LaTeX
% systems. The latest version can be obtained at:
% http://www.ctan.org/tex-archive/macros/latex/contrib/misc/
% Read the url.sty source comments for usage information. Basically,
% \url{my_url_here}.





% *** Do not adjust lengths that control margins, column widths, etc. ***
% *** Do not use packages that alter fonts (such as pslatex).         ***
% There should be no need to do such things with IEEEtran.cls V1.6 and later.
% (Unless specifically asked to do so by the journal or conference you plan
% to submit to, of course. )


% correct bad hyphenation here
\hyphenation{op-tical net-works semi-conduc-tor}






\begin{document}
%
% paper title
% can use linebreaks \\ within to get better formatting as desired
% Do not put math or special symbols in the title.

\title{Modular operator and Holevo's commutation operator}

%
%
% author names and IEEE memberships
% note positions of commas and nonbreaking spaces ( ~ ) LaTeX will not break
% a structure at a ~ so this keeps an author's name from being broken across
% two lines.
% use \thanks{} to gain access to the first footnote area
% a separate \thanks must be used for each paragraph as LaTeX2e's \thanks
% was not built to handle multiple paragraphs
%


 \author{Masaki Sohma
\\
Tamagawa University, Tokyo, Japan}




% If you want to put a publisher's ID mark on the page you can do it like
% this:
%\IEEEpubid{0000--0000/00\$00.00~\copyright~2007 IEEE}
% Remember, if you use this you must call \IEEEpubidadjcol in the second
% column for its text to clear the IEEEpubid mark.



% use for special paper notices
%\IEEEspecialpapernotice{(Invited Paper)}

% make the title area
\maketitle

% As a general rule, do not put math, special symbols or citations
% in the abstract




% Note that keywords are not normally used for peerreview papers.
%\begin{IEEEkeywords}
%IEEEtran, journal, \LaTeX, paper, template.
%\end{IEEEkeywords}






% For peer review papers, you can put extra information on the cover
% page as needed:
% \ifCLASSOPTIONpeerreview
% \begin{center} \bfseries EDICS Category: 3-BBND \end{center}
% \fi
%
% For peerreview papers, this IEEEtran command inserts a page break and
% creates the second title. It will be ignored for other modes.
\IEEEpeerreviewmaketitle


% *** Do not adjust lengths that control margins, column widths, etc. ***
% *** Do not use packages that alter fonts (such as pslatex).         ***
% There should be no need to do such things with IEEEtran.cls V1.6 and later.
% (Unless specifically asked to do so by the journal or conference you plan
% to submit to, of course. )

\bibliographystyle{IEEEtran}


\section{Introduction}
This paper reviews results about a modular operator \cite{Longo_1978} and a Holevo's commutation operator \cite{Holevo_1977} in the simplest case.
We mainly deal with the von Neumann algebra $\mathfrak{N}=M_n(\mathbb{C})$, 
which is the ensemble of bounded linear operators on 
the Hilbert space $\mathfrak{H}=\mathbb{C}^n$ with the inner product $(x,y)=\bar{x}^Ty$. 

Let $\rho$ be a non-degenerated density operator and 
$\omega$ be the corresponding normal state given by $\omega(A)=\mbox{Tr}\rho A$.
Considering an inner product on $\mathfrak{N}$

\begin{equation}
\label{innerP}
\langle A, B \rangle =\frac{1}{2}\omega(BA^{\ast}+A^{\ast}B),
\end{equation}
and a bilinear form on $\mathfrak{N}$
\begin{equation}
[A,B]=i\omega(A^{\ast}B-BA^{\ast}),
\end{equation}
we introduce an operator $\mathfrak{D}$ by 
\begin{equation}\label{Copr}
[A,X]=\langle A, \mathfrak{D}X\rangle.
\end{equation}
The operator $\mathfrak{D}$, called a commutation operator, plays an important role in the quantum estimation theory.  

We regard $\mathfrak{N}$ as a Hilbert-Schmidt space with another inner product
$$
\langle A, B \rangle_2 =\mbox{Tr}(A^{\ast}B),
$$
and denote it by $\mathfrak{H}_2$.
Let us consider a $\ast$-representation on $\mathfrak{H}_2$
\begin{equation}\label{star-rep}
\ell :\mathfrak{N}\to \mathfrak{B}(\mathfrak{H}_2),
\end{equation}
where $\mathfrak{B}(\mathfrak{H}_2)$ is the ensemble of bounded operators on $\mathfrak{H}_2$
and $\ell(A)B=AB$.
Then the state $\omega$ can be written by the inner product $\langle \cdot, \cdot\rangle_2$ as 
$$
\omega(A)=\mbox{Tr}(\rho^{1/2}A\rho^{1/2})=\langle \rho^{1/2},\ell(A)\rho^{1/2}\rangle_2 .
$$
Now we can define the modular operator $\Delta$ for the von Neumann algebra $\mathfrak{M}=\ell(\mathfrak{N})$ and its cyclic separating vector $\rho^{1/2}$. 
We see such derived modular operator is related to the commutation operator
as
$$
    \Delta=\left(1+\frac{i}{2}\mathfrak{D}\right)\left(1-\frac{i}{2}\mathfrak{D}\right)^{-1}.
$$


\section{Representation on $\mathfrak{H}\otimes\mathfrak{H}$}

 We identify the Hilbert-Schmidt space $\mathfrak{H}_2(=\mathfrak{N}=M_n(\mathbb{C}))$ on $\mathfrak{H}=\mathbb{C}^n$ with $\mathfrak{H}\otimes\mathfrak{H}$
by a unitary operator satisfying 
$$
v(e_je_k^T)=e_j\otimes e_k
$$
where $e_j=(\delta_{j1},\delta_{j,2},...,\delta_{j,n})^T$ with the Kronecker delta $\delta_{j,l}$. 
For $\psi=\sum_j \lambda_j e_j$ and $\phi=\sum_k \mu_k e_k$
we have
\begin{equation}
\begin{split}
v(\psi\phi^\ast)&=\sum_{j,k}\lambda_j\bar{\mu}_k v(e_je_k^T)\\
&=\sum_{j,k}\lambda_j\bar{\mu}_k e_j\otimes e_k={\psi}\otimes \overline{\phi}.
\end{split}
\end{equation}

In the $\ast$-representation (\ref{star-rep}), $\ell(A)$ is given by $\tilde{\ell}(A)=A\otimes I_n$ on $\mathfrak{H}\otimes\mathfrak{H}$.
In fact
 \begin{equation}
 \begin{split}
 v(\ell(A)e_je_k^T)&=v(Ae_je_k^T)=Ae_j\otimes e_k\\
 &= (A\otimes I_n)e_k\otimes  e_j=\tilde{\ell}(A)v(e_je_k^T),
 \end{split} 
 \end{equation}
 and hence $\ell(A)=v^{\ast}\tilde{\ell}(A)v$. 
On the other hand, $r(A):\mathfrak{N}\in X \to XA\ni\mathfrak{N}$ is 
given by $\tilde{r}(A)=I_n\otimes A^T$ on $\mathfrak{H}\otimes\mathfrak{H}$.
In fact
 \begin{equation}
 \begin{split}
 v(r(A)e_je_k^T)&=v(e_je_k^TA)=v(e_j(A^\ast e_k)^\ast)=e_j\otimes \overline{A^\ast e_k}\\
 &= e_j \otimes A^T e_k= (I_n\otimes A^T)e_j\otimes  e_k=\tilde{r}(A)v(e_je_k^T),
 \end{split} 
 \end{equation}
 and hence $r(A)=v^{\ast}\tilde{r}(A) v$. 


\section{Modular Operator}
We give a proof  of Tomita-Takesaki theory in the case of 
$\mathfrak{M}=\ell(\mathfrak{N})$ with $\mathfrak{N}=M_n(\mathbb{C})$,
where all difficulties in the theory vanish.
Since
$\tilde{\ell}(\mathfrak{N})^\prime=\tilde{r}(\mathfrak{N})$ and $\tilde{r}(\mathfrak{N})^\prime=\tilde{\ell}(\mathfrak{N})$ for 
$
\tilde{\ell}(\mathfrak{N})=\mathfrak{N}\otimes I_n,\tilde{r}(\mathfrak{N})=I_n\otimes \mathfrak{N},
$
we have
\begin{equation}
\begin{split}
\mathfrak{M}^\prime&=v^\ast \tilde{\ell}(\mathfrak{N})^\prime v= v^\ast \tilde{r}(\mathfrak{N})v=r(\mathfrak{N}) \\
\mathfrak{M}^{\prime\prime}&=v^\ast \tilde{r}(\mathfrak{N})^\prime v= v^\ast \tilde{\ell}(\mathfrak{N})v=\ell(\mathfrak{N})=\mathfrak{M}.
\end{split}
\end{equation}
We use $\rho^{1/2}$ as a cyclic and separating vector;
$$
\mathfrak{H}_2=\mathfrak{M}\rho^{1/2}=\mathfrak{M}^\prime \rho^{1/2}.
$$

 Consider  anti-linear operators on $\mathfrak{H}_2$ 
\begin{equation}\label{SF}
	 \begin{split}
		 S&:\mathfrak{M}\rho^{1/2}\ni \ell(A) \rho^{1/2} \to \ell(A)^\ast\rho^{1/2}\in\mathfrak{M}\rho^{1/2},\\
		 F&:\mathfrak{M}^\prime \rho^{1/2}\ni r(A) \rho^{1/2} \to r(A)^\ast\rho^{1/2}\in \mathfrak{M}^\prime \rho^{1/2}.
	 \end{split}
\end{equation}
	Here 
	$$
	\ell(A)^\ast=v^\ast(A\otimes I_n)^\ast v=v^\ast A^\ast \otimes I_n v=\ell(A^\ast),
	$$
	and
	$$
  r(A)^\ast=v^\ast (I_n\otimes A^T)^\ast v=v^\ast I_n\otimes (A^T)^\ast v=I_n\otimes (A^\ast)^T=r(A^\ast).
	$$
	The linear operator $\Delta=FS$ on $\mathfrak{H}_2$ is called a modular operator.

For the operator $S$, we have 
	$$
    S(X)=\rho^{-1/2}X^*\rho^{1/2}.
 	$$
	In fact, putting $X=\ell(A)\rho^{1/2}=A\rho^{1/2}$,$Y=\ell(A)^\ast\rho^{1/2}=\ell(A^\ast)\rho^{1/2}=A^\ast \rho^{1/2}$,
	$$
    Y=(X\rho^{-1/2})^\ast \rho^{1/2}=\rho^{-1/2}X^\ast\rho^{1/2},
  $$
On the other hand, for the operator $F$, we have
$$
F(X)=\rho^{1/2}X^\ast \rho^{-1/2},
$$
In fact, putting $X=r(A)\rho^{1/2}=\rho^{1/2}A$,$Y=r(A)^\ast\rho^{1/2}=r(A^\ast)\rho^{1/2}= \rho^{1/2}A^\ast$,
	$$
    Y= \rho^{1/2}(\rho^{-1/2}X)^\ast=\rho^{1/2}X^\ast\rho^{-1/2}.
  $$
Thus
$$
\Delta(X)=FS(X)=F(\rho^{-1/2}X^*\rho^{1/2})=\rho^{1/2}(\rho^{-1/2}X^*\rho^{1/2})^\ast \rho^{-1/2}=\rho X \rho^{-1},
$$
that is, $\Delta=v^\ast(\rho \otimes ({\rho}^{-1})^T) v$.
It follows that $\Delta^\ast=v^\ast (\rho\otimes({\rho}^{-1})^T)^\ast v=\Delta$.
Since $\Delta^{-1/2}=v^\ast(\rho^{-1/2}\otimes ({\rho}^{1/2})^T) v$, 
$$
\Delta^{-1/2}(X)=\rho^{-1/2}X\rho^{1/2}
$$
and hence 
$$
S(X)=\Delta^{-1/2}(X^\ast)=\Delta^{-1/2}J(X),
$$
where $J$ is an anti-linear operator defined by $J(X)=X^\ast$.
In a similar way we have
$$
F(X)=\Delta^{1/2}J(X).
$$
In addition 
$$
J\Delta J(X)=(\rho X^\ast \rho^{-1})^\ast=\rho^{-1}X\rho=\Delta^{-1}(X).
$$

We have 
$$
\Delta^{-it}\ell(A)\Delta^{it}=\ell(\rho^{-it}A\rho^{it}).
$$
Indeed $$\Delta^{-it}\ell(A)\Delta^{it}(X)=\Delta^{-it}(A\rho^{it}X\rho^{-it})=\rho^{-it}A\rho^{it}X\rho^{-it}\rho^{it}=\rho^{-it}A\rho^{it}X$$
and 
$$
\ell(\rho^{-it}A\rho^{it})X=\rho^{-it}A\rho^{it}X.
$$
Thus
$$
\Delta^{-it}\mathfrak{M}\Delta^{it}=\mathfrak{M}.
$$
     Finally we have 
       $$
          J\mathfrak{M}J=\mathfrak{M}^{\prime},
       $$
			 which is obtained from
			 $$
J\ell(A)J(X)=J(AX^\ast)=(AX^\ast)^\ast=XA^\ast
			 $$
			 and
			 $$
r(A)X=XA.
			 $$
Remark that the above discussion can be easily extend to the case where 
$\mathfrak{N}=M_{m_1}(\mathbb{C})\oplus \cdots \oplus M_{m_n}(\mathbb{C})$.
The proof of Tomita-Takesaki theory for a finite dimensional von Neumann algebra is given in the Appendix.


\section{Relation between Holevo's commutation operator and modular operator}

Let us see how the modular operator defined by  (\ref{Copr}) is described on $\mathfrak{H}\otimes \mathfrak{H}$.
Since it holds for $A,X,Y=\mathfrak{D}X\in \mathfrak{H}(=\mathfrak{N}=\mathfrak{H}_2)$
that 
 \begin{equation}
 \begin{split}
 [A,X]&=i\omega(A^\ast X-XA^\ast)=i\mbox{Tr}\rho(A^\ast X-XA^\ast)\\
      &=\mbox{Tr}A^\ast i(X\rho-\rho X)=\langle A, i(X\rho-\rho X)\rangle_2\\
 \langle A,Y\rangle&=\omega((YA^\ast+A^\ast Y)/2)=\mbox{Tr}\rho((YA^\ast+A^\ast Y)/2)\\
 &=\mbox{Tr}A^\ast(\rho Y+Y\rho)/2=\langle A, \rho Y+Y\rho \rangle_2,
 \end{split}
 \end{equation}
  we have
 $$
 (\rho Y+Y\rho)/2=i(X\rho-\rho X),
 $$
 which can be represented on $\mathfrak{H}\otimes \mathfrak{H}$ as
	$$
	(I_n\otimes \rho + \rho\otimes I_n)v(Y)=2i(\rho\otimes I_n -I_n\otimes \rho )v(X).
	$$
	Thus 
	$$
	v(\mathfrak{D}X)=v(Y)=2i(I_n\otimes \rho + \rho\otimes I_n)^{-1}(\rho\otimes I_n -I_n\otimes \rho )v(X),
	$$
	that is,
	$$
	\mathfrak{D}=v^\ast[ 2i(I_n\otimes \rho + \rho\otimes I_n)^{-1}(\rho\otimes I_n -I_n\otimes \rho )]v
	$$
	  Moreover
	\begin{equation}
		\begin{split}
    1+\frac{i}{2}\mathfrak{D}&=v^\ast [ 2(I_n\otimes \rho+\rho\otimes I_n)^{-1}  I_n\otimes \rho]v,\\
    1-\frac{i}{2}\mathfrak{D}&=v^\ast [ 2(I_n\otimes \rho+\rho\otimes I_n)^{-1}\rho \otimes I_n]v,\\
    1+\frac{1}{4}\mathfrak{D}^2&=v^\ast[ 4(I_n\otimes \rho+\rho\otimes I_n)^{-2}\rho \otimes \rho]v.\\
    \end{split}
	\end{equation}

	Finally	we get
		$$
    \Delta=\left(1+\frac{i}{2}\mathfrak{D}\right)\left(1-\frac{i}{2}\mathfrak{D}\right)^{-1},
		$$
		and
		$$
     \frac{i}{2}\mathfrak{D}=(1-\Delta)(1+\Delta)^{-1}
		$$



\section{Appendix}
In this section we prove the Tomita-Takesaki theorem for a finite dimensional
von Neumann algebras $\tilde{\mathfrak{N}}$ on a Hilbert space $\mathfrak{K}$.
We assume there exists a cyclic separating vector $\xi\in \mathfrak{K}$;
$\mathfrak{K}=\tilde{\mathfrak{N}}\xi=\tilde{\mathfrak{N}}^\prime\xi$.
Then we define operators $\tilde{S},\tilde{F},\tilde{\Delta}$ on $\mathfrak{K}$ as
\begin{equation}
\begin{split}
\tilde{S}&:\tilde{\mathfrak{N}}\xi\ni X\xi \to X^\ast \xi \in \tilde{\mathfrak{N}}\xi,\\
\tilde{F}&:\tilde{\mathfrak{N}}^\prime\xi\ni Y\xi \to Y^\ast \xi \in \tilde{\mathfrak{N}}^\prime\xi,\\
\tilde{\Delta}&=\tilde{F}\tilde{S}.
\end{split}
\end{equation}
The Wedderburn theorem states that a finite dimensional $C^{\ast}$ algebra is  $\ast$-isomorphic to a direct sum of simple matrix algebras. That is, there exists $\ast$-isomorphic function $\varphi$ for von Neumann algebra $\mathfrak{N}$ such that
$$
\varphi:\tilde{\mathfrak{N}}\simeq \mathfrak{N}:=M_{m_1}(\mathbb{C})\oplus \cdots \oplus M_{m_n}(\mathbb{C})
$$

Let us consider the state on $\tilde{\mathfrak{N}}$ as
$$
\omega_\xi(A)=(\xi,A\xi)_{\mathfrak{K}},\quad A\in \tilde{\mathfrak{N}},\xi\in \mathfrak{K}.
$$
Using this state we can define the state on $\mathfrak{N}$
$$
\omega(A)=\omega_\xi(\varphi^{-1}(A)),A\in \mathfrak{N}.
$$
Applying the discussion in the previous sections to the von Neumann algebra $\mathfrak{N}$ and 
$\omega$, we get
the von Neumann algebra $\mathfrak{M}=\ell (\mathfrak{N})$ on the Hilbert space 
$\mathfrak{H}_2$, the cyclic separating vector $\rho^{1/2}\in \mathfrak{H}_2$
and operators $S, F$ and $\Delta$ on $\mathfrak{H}_2$.

Then we obtain
$$
\langle \pi(A)\rho^{1/2},\pi( B)\rho^{1/2}\rangle_2=(\varphi^{-1}(A)\xi,\varphi^{-1}(B)\xi)_\mathfrak{K},
$$
which means
$$
U:\mathfrak{H}_2\ni\pi(A)\rho^{1/2}\to \varphi^{-1}(A)\xi\in\mathfrak{K}, \quad A\in \mathfrak{N}
$$
gives a unitary operator from $\mathfrak{H}_2$ to $\mathfrak{K}$.
This unitary operator combines $\Delta$ and $\tilde{\Delta}$ as
$$
\tilde{\Delta}=U\Delta U^{\ast},
$$
and gives the main result of Tomita-Takesaki Theory,
$$
\tilde{\Delta}^{-it}\tilde{\mathfrak{N}}\tilde{\Delta}^{it}=\tilde{\mathfrak{N}}.
$$

\bibliography{bibliography/converted_to_latex.bib%
}

\end{document}

