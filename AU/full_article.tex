\documentclass{article}
\usepackage[affil-it]{authblk}
\usepackage{graphicx}
\usepackage[space]{grffile}
\usepackage{latexsym}
\usepackage{textcomp}
\usepackage{longtable}
\usepackage{multirow,booktabs}
\usepackage{amsfonts,amsmath,amssymb}
\usepackage{natbib}
\usepackage{url}
\usepackage{hyperref}
\hypersetup{colorlinks=false,pdfborder={0 0 0}}
% You can conditionalize code for latexml or normal latex using this.
\newif\iflatexml\latexmlfalse
\providecommand{\tightlist}{\setlength{\itemsep}{0pt}\setlength{\parskip}{0pt}}%

\usepackage[utf8]{inputenc}
\usepackage[english]{babel}




\begin{document}

\title{Modular operator and Holevo's commutation operator}


 \author{Masaki Sohma
\\
Tamagawa University, Tokyo, Japan}


\date{\today}

\bibliographystyle{plain}

\maketitle




\section{Introduction}
This paper reviews results about  modular operators \cite{Longo_1978} and  commutation operators \cite{Holevo_1977} in the simplest case.
We mainly deal with the von Neumann algebra $\mathfrak{N}=M_n(\mathbb{C})$, 
which is the ensemble of $n\times n$ complex matrices and can be considered as the algebra $\mathfrak{B}({\cal H})$ of bounded linear operators on 
the Hilbert space ${\cal H}=\mathbb{C}^{n}$ with the inner product $(x,y)=\bar{x}^Ty$. 
Let $\rho$ be a non-degenerated density operator and 
$\omega$ be the corresponding normal state given by $\omega(A)=\mbox{Tr}\rho A, A\in \mathfrak{N}$.
We regard $\mathfrak{N}$ as a Hilbert space with the inner product
\begin{equation}
\label{innerP}
\langle A, B \rangle =\frac{1}{2}\omega(BA^{\ast}+A^{\ast}B),
\end{equation}
and denote it by $\mathfrak{H}$.


In the quantum theory we consider additional bilinear form on $\mathfrak{H}$ 
\begin{equation}\label{Bform}
[A,B]=i\omega(A^{\ast}B-BA^{\ast}).
\end{equation}
and  obtain fundamental inequalities 
$$
\langle X, X\rangle \geq \pm \frac{i}{2}[X,X],
$$
which yield  the uncertainty relation of the most general form \cite{Holevo_1977}.
We define a commutation operator $\mathfrak{D}$ so that it satisfies 
\begin{equation}\label{Copr}
[A,X]=\langle A, \mathfrak{D}X\rangle.
\end{equation}
The operator $\mathfrak{D}$, firstly introduced by Holevo \cite{Holevo_1977}, plays an important role in the non-commutative statistical theory. From Eq. (\ref{Bform}) it holds that 
\begin{equation}
1\pm \frac{i}{2}\mathfrak{D}\geq 0.
\end{equation}



On the other hand we can also regard $\mathfrak{N}$ as a Hilbert-Schmidt space with another inner product
$$
\langle A, B \rangle_2 =\mbox{Tr}(A^{\ast}B),
$$
by virtue of its finite-dimensionality and denote it by $\mathfrak{H}_2$.
Let us consider a $\ast$-representation on $\mathfrak{H}_2$
\begin{equation}\label{star-rep}
\ell :\mathfrak{N}\to \mathfrak{B}(\mathfrak{H}_2),
\end{equation}
where $\mathfrak{B}(\mathfrak{H}_2)$ is the ensemble of bounded operators on $\mathfrak{H}_2$
and $\ell(A)B=AB$.
Then the state $\omega$ can be written by the inner product $\langle \cdot, \cdot\rangle_2$ as 
$$
\omega(A)=\mbox{Tr}(\rho^{1/2}A\rho^{1/2})=\langle \rho^{1/2},\ell(A)\rho^{1/2}\rangle_2 .
$$
%Note that the two norms induced from the inner products $\langle \cdot,\cdot \rangle$ and 
%$\langle \cdot,\cdot \rangle_2$ are related as
%\begin{equation}
%\begin{split}
%\parallel A\parallel^2 &= \langle A,A\rangle \\
%&=\frac{1}{2}(\langle \rho^{1/2}A,\rho^{1/2}A\rangle_2+\langle %A\rho^{1/2},A\rho^{1/2}\rangle_2  )\\
%&\leq \frac{1}{2}(\parallel \rho^{1/2}A\parallel_2^2+\parallel A\rho^{1/2}\parallel_2^2)\\
%&=\parallel \rho^{1/2}\parallel_{\cal H} ^2 \parallel A\parallel_2^2,
%\end{split}
%\end{equation}
%where $\parallel \cdot \parallel_{\cal H}$ is the operator norm for $\mathfrak{B}({\cal H})$
%and $\parallel \cdot \parallel_2$ is the norm for the Hilbert-Schdmit space %$\mathfrak{H}_2$. 


In the present case we have $\mathfrak{H}_2=\mathfrak{H}$, but when we consider an infinite
dimensional Hilbert space ${\cal H}$ the equality does not hold, i.e. $\mathfrak{H}_2\subset \mathfrak{B}({\cal H})\subset \mathfrak{H}$. This makes it difficult to extend the discussion in Sec. 4 to the infinite dimensional case.

As stated in Sec. 3, we can define the
modular operator $\Delta$ for the von Neumann algebra $\mathfrak{M}=\ell(\mathfrak{N})$ and its cyclic separating vector $\rho^{1/2} \in \mathfrak{H}_2$. 
In this paper we derive a simple relation between such derived modular operator and the commutation operator:
$$
    \Delta=\left(1+\frac{i}{2}\mathfrak{D}\right)\left(1-\frac{i}{2}\mathfrak{D}\right)^{-1},
$$
which is originally shown in \cite{Holevo_1977}.

\section{Representation on $\cal{H}\otimes\cal{H}$}

 We identify the Hilbert-Schmidt space $\mathfrak{H}_2(=\mathfrak{N}=M_n(\mathbb{C}))$ on ${\cal H}=\mathbb{C}^n$ with $\cal{H}\otimes\cal{H}$
by a unitary operator satisfying 
$$
v(e_je_k^T)=e_j\otimes e_k,
$$
where $e_j=(\delta_{j1},\delta_{j2},...,\delta_{jn})^T$ with the Kronecker delta $\delta_{jl}$. 
Here, for $\psi=\sum_j \lambda_j e_j$ and $\phi=\sum_k \mu_k e_k$
we have
\begin{equation}
\begin{split}
v(\psi\phi^\ast)&=\sum_{j,k}\lambda_j\bar{\mu}_k v(e_je_k^T)\\
&=\sum_{j,k}\lambda_j\bar{\mu}_k e_j\otimes e_k={\psi}\otimes \overline{\phi}.
\end{split}
\end{equation}
In the $\ast$-representation (\ref{star-rep}), $\ell(A)$ is given by $\tilde{\ell}(A)=A\otimes I_n$ on $\cal{H}\otimes\cal{H}$.
In fact
 \begin{equation}
 \begin{split}
 v(\ell(A)e_je_k^T)&=v(Ae_je_k^T)=Ae_j\otimes e_k\\
 &= (A\otimes I_n)e_k\otimes  e_j=\tilde{\ell}(A)v(e_je_k^T),
 \end{split} 
 \end{equation}
 and hence $\ell(A)=v^{\ast}\tilde{\ell}(A)v$. 


On the other hand, 
$$
r(A):\mathfrak{N}\in X \to XA\ni\mathfrak{N}
$$ 
is represented by $\tilde{r}(A)=I_n\otimes A^T$ on $\cal{H}\otimes\cal{H}$.
In fact
 \begin{equation}
 \begin{split}
 v(r(A)e_je_k^T)&=v(e_je_k^TA)=v(e_j(A^\ast e_k)^\ast)=e_j\otimes \overline{A^\ast e_k}\\
 &= e_j \otimes A^T e_k= (I_n\otimes A^T)e_j\otimes  e_k=\tilde{r}(A)v(e_je_k^T),
 \end{split} 
 \end{equation}
 and hence $r(A)=v^{\ast}\tilde{r}(A) v$. 

Remark that $\tilde{\ell}(\mathfrak{N})=\mathfrak{N}\otimes I_n$ and $\tilde{r}(\mathfrak{N})=I_n\otimes \mathfrak{N}$ are von Neumann algebras in 
$$\mathfrak{B}({\cal H}\otimes{\cal H})=\mathfrak{B}({\cal H})\otimes \mathfrak{B}({\cal H})=\mathfrak{N}\otimes \mathfrak{N},
$$
and we have 
\begin{equation}\label{lr}
\tilde{\ell}(\mathfrak{N})^\prime=\tilde{r}({\mathfrak{N}}),\quad
\tilde{r}(\mathfrak{N})^\prime=\tilde{\ell}(\mathfrak{N}).
\end{equation}

\section{Modular Operator}
We give a proof  of the main results of Tomita-Takesaki theory in the case of 
$\mathfrak{M}=\ell(\mathfrak{N}) \subset \mathfrak{B}(\mathfrak{H}_2)$ with $\mathfrak{N}=M_n(\mathbb{C})$,
where all difficulties in the theory vanish.
From Eq. (\ref{lr}),
we have
\begin{equation}
\begin{split}
\mathfrak{M}^\prime&=v^\ast \tilde{\ell}(\mathfrak{N})^\prime v= v^\ast \tilde{r}(\mathfrak{N})v=r(\mathfrak{N}) \\
\mathfrak{M}^{\prime\prime}&=v^\ast \tilde{r}(\mathfrak{N})^\prime v= v^\ast \tilde{\ell}(\mathfrak{N})v=\ell(\mathfrak{N})=\mathfrak{M}.
\end{split}
\end{equation}
We introduce  a cyclic and separating vector $\rho^{1/2}$, satisfying 
$$
\mathfrak{H}_2=\mathfrak{M}\rho^{1/2}=\mathfrak{M}^\prime \rho^{1/2},
$$
and consider  anti-linear operators on $\mathfrak{H}_2$ 
\begin{equation}\label{SF}
	 \begin{split}
		 S&:\ell(A) \rho^{1/2} \to \ell(A)^\ast\rho^{1/2},\\
		 F&:r(A) \rho^{1/2} \to r(A)^\ast\rho^{1/2}.
	 \end{split}
\end{equation}


	Here 
	$$
	\ell(A)^\ast=v^\ast(A\otimes I_n)^\ast v=v^\ast A^\ast \otimes I_n v=\ell(A^\ast),
	$$
	and
	$$
  r(A)^\ast=v^\ast (I_n\otimes A^T)^\ast v=v^\ast I_n\otimes (A^T)^\ast v=I_n\otimes (A^\ast)^T=r(A^\ast).
	$$
	The linear operator $\Delta=FS$ on $\mathfrak{H}_2$ is known as a modular operator.

For the operator $S$, we have 
	$$
    S(X)=\rho^{-1/2}X^*\rho^{1/2}.
 	$$
	In fact, putting $X=\ell(A)\rho^{1/2}=A\rho^{1/2}$,$Y=\ell(A)^\ast\rho^{1/2}=\ell(A^\ast)\rho^{1/2}=A^\ast \rho^{1/2}$,
	$$
    Y=(X\rho^{-1/2})^\ast \rho^{1/2}=\rho^{-1/2}X^\ast\rho^{1/2}.
  $$
On the other hand, for the operator $F$, we have
$$
F(X)=\rho^{1/2}X^\ast \rho^{-1/2}.
$$
In fact, putting $X=r(A)\rho^{1/2}=\rho^{1/2}A$,$Y=r(A)^\ast\rho^{1/2}=r(A^\ast)\rho^{1/2}= \rho^{1/2}A^\ast$,
	$$
    Y= \rho^{1/2}(\rho^{-1/2}X)^\ast=\rho^{1/2}X^\ast\rho^{-1/2}.
  $$


Thus
$$
\Delta(X)=FS(X)=F(\rho^{-1/2}X^*\rho^{1/2})=\rho^{1/2}(\rho^{-1/2}X^*\rho^{1/2})^\ast \rho^{-1/2}=\rho X \rho^{-1},
$$
that is, 
\begin{equation}\label{Delta}
\Delta=v^\ast(\rho \otimes ({\rho}^{-1})^T) v.
\end{equation}
It follows that $\Delta^\ast=v^\ast (\rho\otimes({\rho}^{-1})^T)^\ast v=\Delta$.
Since $\Delta^{-1/2}=v^\ast(\rho^{-1/2}\otimes ({\rho}^{1/2})^T) v$, 
$$
\Delta^{-1/2}(X)=\rho^{-1/2}X\rho^{1/2}
$$
and hence 
$$
S(X)=\Delta^{-1/2}(X^\ast)=\Delta^{-1/2}J(X),
$$
where $J$ is an anti-linear operator defined by $J(X)=X^\ast$.
In a similar way we have
$$
F(X)=\Delta^{1/2}J(X).
$$


Since
\begin{equation}
\begin{split}
\Delta^{-it}\ell(A)\Delta^{it}(X)&=\Delta^{-it}(A\rho^{it}X\rho^{-it})=\rho^{-it}A\rho^{it}X\rho^{-it}\rho^{it}\\
                                 &=\rho^{-it}A\rho^{it}X=\ell (\rho^{-it}A\rho^{it})(X),
\end{split}
\end{equation}
we have 
$$
\Delta^{-it}\ell(A)\Delta^{it}=\ell(\rho^{-it}A\rho^{it}).
$$
On the other hand,
we have
$$
J\ell(A)J(X)=J(AX^\ast)=(AX^\ast)^\ast=XA^\ast=r(A^\ast)(X)
$$
Thus we obtain the main results of Tomita-Takesaki theory in our case: 
\begin{equation}
\begin{split}
\Delta^{-it}\mathfrak{M}\Delta^{it}=\mathfrak{M},\\
          J\mathfrak{M}J=\mathfrak{M}^{\prime}.
\end{split}
\end{equation}
Remark that the above discussion can be easily extend to the case where 
$\mathfrak{N}=M_{m_1}(\mathbb{C})\oplus \cdots \oplus M_{m_n}(\mathbb{C})$.
The proof of Tomita-Takesaki theory for a finite dimensional von Neumann algebra is given in the Appendix.





\section{Relation between Holevo's commutation operator and modular operator}

Let us see how the modular operator defined by  (\ref{Copr}) is described on $\cal{H}\otimes \cal{H}$.
Since it holds for $A,X,Y=\mathfrak{D}X\in \mathfrak{H}(=\mathfrak{N}=\mathfrak{H}_2)$
that 
 \begin{equation}
 \begin{split}
 [A,X]&=i\omega(A^\ast X-XA^\ast)=i\mbox{Tr}\rho(A^\ast X-XA^\ast)\\
      &=\mbox{Tr}A^\ast i(X\rho-\rho X)=\langle A, i(X\rho-\rho X)\rangle_2\\
 \langle A,Y\rangle&=\omega((YA^\ast+A^\ast Y)/2)=\mbox{Tr}\rho((YA^\ast+A^\ast Y)/2)\\
 &=\mbox{Tr}A^\ast(\rho Y+Y\rho)/2=\langle A, \rho Y+Y\rho \rangle_2,
 \end{split}
 \end{equation}
  we have
 $$
 (\rho Y+Y\rho)/2=i(X\rho-\rho X),
 $$
 which can be represented on $\cal{H}\otimes \cal{H}$ as
	$$
	(\rho\otimes I_n+I_n\otimes \rho^T )v(Y)=2i(I_n\otimes \rho^T-\rho\otimes I_n)v(X).
	$$
	Thus 
	$$
	v(\mathfrak{D}X)=v(Y)=2i(\rho\otimes I_n+I_n\otimes \rho^T )^{-1}(I_n\otimes \rho^T -\rho\otimes I_n )v(X),
	$$
	that is,
	$$
	\mathfrak{D}=v^\ast[ 2i( \rho\otimes I_n+I_n\otimes \rho^T )^{-1}(I_n\otimes \rho^T -\rho\otimes I_n )]v.
	$$
	 

 Moreover
	\begin{equation}  
		\begin{split}
    1+\frac{i}{2}\mathfrak{D}&=v^\ast [ 2(\rho\otimes I_n+I_n\otimes \rho)^{-1}  \rho \otimes I_n]v,\\
    1-\frac{i}{2}\mathfrak{D}&=v^\ast [ 2(\rho\otimes I_n+I_n\otimes \rho)^{-1}I_n \otimes \rho^T]v,\\
    1+\frac{1}{4}\mathfrak{D}^2&=v^\ast[ 4(\rho\otimes I_n+I_n\otimes \rho)^{-2}\rho \otimes \rho^T]v.\\
    \end{split}
	\end{equation}
Using these equations and Eq. (\ref{Delta})  we conclude
		$$
    \Delta=\left(1+\frac{i}{2}\mathfrak{D}\right)\left(1-\frac{i}{2}\mathfrak{D}\right)^{-1},
		$$
		and
		$$
     \frac{i}{2}\mathfrak{D}=(\Delta-1)(\Delta+1)^{-1}.
		$$




\section{Appendix}
In this section we prove the main results of Tomita-Takesaki theorem for a finite dimensional
von Neumann algebras $\tilde{\mathfrak{N}}$ on a Hilbert space $\cal{K}$.
We assume there exists a cyclic separating vector $\xi\in \cal{K}$;
$\cal{K}=\tilde{\mathfrak{N}}\xi=\tilde{\mathfrak{N}}^\prime\xi$.
Then we define operators $\tilde{S},\tilde{F},\tilde{\Delta}$ and $\tilde{J}$ on $\cal{K}$ as
\begin{equation}
\begin{split}
\tilde{S}&:X\xi \to X^\ast \xi \\
\tilde{F}&:Y\xi \to Y^\ast \xi \\
\tilde{\Delta}&=\tilde{F}\tilde{S},\\
\tilde{J}&=\tilde{\Delta}^{1/2}\tilde{S}.
\end{split}
\end{equation}
The Wedderburn theorem states that a finite dimensional $C^{\ast}$ algebra is  $\ast$-isomorphic to a direct sum of simple matrix algebras. That is, there exists $\ast$-isomorphic function $\varphi$ for von Neumann algebra $\tilde{\mathfrak{N}}$ such that
$$
\varphi:\tilde{\mathfrak{N}}\simeq \mathfrak{N}:=M_{m_1}(\mathbb{C})\oplus \cdots \oplus M_{m_n}(\mathbb{C}).
$$


Let us consider the state on $\tilde{\mathfrak{N}}$ as
$$
\omega_\xi(A)=(\xi,A\xi)_{\cal K},\quad A\in \tilde{\mathfrak{N}},\xi\in \cal{K},
$$
where $(\cdot,\cdot)_{\cal K}$ is an inner product of the Hilbert space ${\cal K}$.
Using this state we can define the state on $\mathfrak{N}$
$$
\omega(A)=\omega_\xi(\varphi^{-1}(A)),A\in \mathfrak{N},
$$
which is normal by virtue of finite-dimensionality.
Applying the discussion in the previous sections to the von Neumann algebra $\mathfrak{N}$ and the state
$\omega$, we get
the von Neumann algebra $\mathfrak{M}=\ell (\mathfrak{N})$ on  
$\mathfrak{H}_2$, the cyclic separating vector $\rho^{1/2}\in \mathfrak{H}_2$
and the operators on $\mathfrak{H}_2$, $S, F, J$ and $\Delta$.
In particular we have
$$
\langle \ell(A)\rho^{1/2},\ell( B)\rho^{1/2}\rangle_2=(\varphi^{-1}(A)\xi,\varphi^{-1}(B)\xi)_{\cal K},
$$
which means
$$
U:\mathfrak{H}_2\ni\ell(A)\rho^{1/2}\to \varphi^{-1}(A)\xi\in {\cal K}, \quad A\in \mathfrak{N}
$$
gives a unitary operator from $\mathfrak{H}_2$ to ${\cal K}$.
Using this unitary operator we obtain the following relations
\begin{equation}
\begin{split}
\tilde{S}&=US U^{\ast},\\
\tilde{F}&=UFU^\ast ,
\end{split}
\end{equation}
and hence $\tilde{\Delta}=U\Delta U^\ast$, $\tilde{J}=UJU^\ast$.
Moreover  
$$
\tilde{\mathfrak{N}}=U\ell(\mathfrak{N})U^\ast,
$$
since $\varphi^{-1}(X)=U\ell (X) U^\ast$ for $X\in \mathfrak{N}$.
Thus we conclude the main result of Tomita-Takesaki Theory,
\begin{equation}
\begin{split}
\tilde{\Delta}^{-it}\tilde{\mathfrak{N}}\tilde{\Delta}^{it}=\tilde{\mathfrak{N}}\\
\tilde{J}\tilde{\mathfrak{N}}\tilde{J}=\tilde{\mathfrak{N}}^\prime.
\end{split}
\end{equation}





\paragraph{Acknowledgement}\label{acknowledgement}

I am grateful to A. Ohashi, T. Sogabe and T. Usuda for helpful
discussions.

\bibliography{bibliography/converted_to_latex.bib%
}

\end{document}

